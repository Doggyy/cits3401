\documentclass[12pt, a4paper]{article}

\usepackage{titling}

\newcommand{\subtitle}[1]{%
  \posttitle{%
    \par\end{center}
    \begin{center}\large#1\end{center}
    \vskip0.5em}%
}

\newcommand{\specialcell}[2][c]{%
  \begin{tabular}[#1]{@{}l@{}}#2\end{tabular}}
  
\setlength{\oddsidemargin}{0.5cm}
\setlength{\evensidemargin}{0.5cm}
\setlength{\topmargin}{-1.6cm}
\setlength{\leftmargin}{0.5cm}
\setlength{\rightmargin}{0.5cm}
\setlength{\textheight}{24.00cm} 
\setlength{\textwidth}{15.00cm}
\parindent 0pt
\parskip 5pt
\pagestyle{plain}
\usepackage[numbers]{natbib}
\usepackage{capt-of}
\usepackage{graphicx,url}

\newcommand{\namelistlabel}[1]{\mbox{#1}\hfil}
\newenvironment{namelist}[1]{%1
\begin{list}{}
    {
        \let\makelabel\namelistlabel
        \settowidth{\labelwidth}{#1}
        \setlength{\leftmargin}{1.1\labelwidth}
    }
  }{%1
\end{list}}

\begin{document}

\title{CITS3401 Data Exploration \& Mining Project 1}
\subtitle{Healthy Burgers Fast Food Chain}
\author{Aleck Greenham 20362627 \\ Ash Tyndall 20915779}
\date{26th April 2013}

\maketitle

\section*{Abstract}

This document details the design of a data cube suitable for the \textit{Healthy Burgers} franchise, a fast-food restaurant chain with stores in three countries, to aid management in its plans for improving sales and global expansion. Where the system requirements \cite{designdoc} were ambiguous or incomplete, reasonable assumptions were made and documented.

\section*{Introduction}

The client, Healthy Burgers, wishes to mine historical sales data from their nine stores spread across three countries across Australasia to aid management in making decisions regarding global expansion of the franchise and improving sales at existing restaurants. Each restaurant maintains a Online Transaction Processing (OLTP) database and there are aggregate databases at the state, country and worldwide levels. The company maintains data on sales, products, suppliers and restaurants. The purpose of this document is to detail the design of a data warehouse to extract data from these databases that will support the decisions faced by Healthy Burgers management.

\section*{Limitations}

It is outside the scope of this assessment to address the importing of data from the heterogeneous OLTP restaurant databases, including the cleaning of data. It shall be assumed that the data is complete, easily available and in the desired formats. It is acknowledged, however, that this would not be the case in real industry applications.

A detailed interpretation of the results of the OLAP operations is also outside the scope of this assessment. It falls to Healthy Burgers management to fully interpret and make recommendations or decisions based on the data stored within the data warehouse.

\section*{Requirements}

Below is a summary of authors' interpretation of the initial requirements:

\begin{tabular}{|l|l|l|}

\hline
\textbf{Object} & \textbf{Properties} & \textbf{Restrictions} \\
\hline

Country & & 
	Australia, New Zealand, Singapore 
	\\
\hline

Store & 
	\specialcell{Interior Design \\ Facility Type} & 
	\specialcell{3 restaurants in each country \\ 3 different interior designs \\ Facilities are 'dine in', 'drive through' and 'both'} 
	\\ 	
\hline

Combo Meal &
	\specialcell{Price Category \\ Calorie Value} & 
	\specialcell{6 different combo meals \\ Not all stores have combo meals \\ Divided into 3 different price categories}
	\\
\hline

Supplier &
	&
	\specialcell{There are 2 suppliers \\ Suppliers do 3 combo meals each}
	\\
\hline

Sale &
	\specialcell{Date \\ Product} &
	Date Between 2008 and 2012
	\\
\hline

Sales Period &
	&
	\specialcell{Has 2 combo meals \\ Can be 'breakfast', 'lunch' or 'dinner'}
	\\
\hline	
\end{tabular}

\section*{Assumptions}

The following assumptions were made to:

\begin{itemize}
	\item To make the initial requirements \cite{designdoc} sufficiently complete
	\item To make the design scenario as realistic as possible 
	\item For the sake of simplicity or clarity, where realism was impossible, impractical or outside the scope of assessment
\end{itemize}

\begin{enumerate}
	\item The type of ingredients are not important, only their supplier - i.e. the supplier is a proxy for how desirable the ingredients they supply are. 
	\item Sale periods are identified by their names and not by time ranges.
	\item There are six combo meals globally.
	\item There are 3 sales periods, each with 2 combo meals, and there only being a total of 6 combo meals to choose from. If it’s true that not all stores sell all the combo meals, then some combo meals must be shared between sales periods.
	\item All non-combo products sold are not of interest (because they have no relation with any of the other data fields).
	\item The same suppliers supply the ingredients for same combo meals, globally.
	\item A promotional period has it’s own 6 combo meals (thus the combo meals change every 2 months).
	\item The price of combo meals is set at the store level.
\end{enumerate}

\begin{thebibliography}{9}

\bibitem{designdoc}
	CITS3401 Data Exploration and Mining - Project 1,
	http://undergraduate.csse.uwa.edu.au/units/CITS3401/labs/proj1-2013.html
		
\end{thebibliography}

\bibliographystyle{IEEEtranN}

\end{document}
